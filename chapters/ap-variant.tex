\chapter{Fabric Variants}
\label{sec:append:variants}
\begin{table}[h]
    \centering
    \caption{The naming conventions of all Fabric variants benchmarked}
    \label{append:variant}
\begin{tabular}{|p{.4\textwidth}||*{3}{c|}}\hline
    \backslashbox[.43\textwidth]{Concurrency Control}{State Storage}
    & LevelDB & \color{blue}
    LevelDBProv & \color{blue}ForkBase
    \\\hline\hline

    The original First-in-first-out~\cite{github:fabric} &\textbf{Fabric}&\textbf{{\fsPrO}}& \textbf{{\fsO}}\\\hline
    The approach from Fabric++~\cite{sharma2019blurring} &\na & \na &  \textbf{\fsP}\\\hline
    \color{purple}The standard OCC technique in databases~\cite{CahillRF08}  & \na & \na & \textbf{\fsS}\\\hline
    \color{purple}The latest OCC technique recently published in a database literature~\cite{ding2018improving} &\na& \na &\textbf{\fsL} \\\hline
    \color{purple}Ours (proposed in Chapter~\ref{ch:txn}) &\na& \textbf{\fsPrF}& \textbf{\fsF}\\
    \hline
\end{tabular}
\label{tab:append:variant}
\end{table}
Table~\ref{tab:append:variant} compiles all Fabric variants benchmarked in the thesis. 
They can be classified into two dimensions, the storage engine and the concurrency control methods. 
The three concurrency control methods colored in \textcolor{purple}{purple} required for the snapshot simulation described in Chapter~\ref{sec:txn:impl:snapshot_read}. The storage engines in \textcolor{blue}blue support the data provenance, while the original LevelDB only maintains the latest states.
Chapter~\ref{sec:provenance:exp:setup} explains how we extend on LevelDB into LevelDBProv, which manages the historical data and the dependency (provenance).
Apart from the above, 
{\fabricPlusplus}~\cite{sharma2019blurring} is another Fabric variant purposed to optimize on the concurrency control. 
{\fabricPlusplus} is implemented on Fabric v1.1. 
We adapt their techniques to Fabric v2.2, which results into {\fsP}. 
{\ff} is featured for the platform-specific optimization on Fabric v1.2, which is orthogonal to ours. 