% \SetPicSubDir{ch-Rice}
% \SetExpSubDir{ch-Rice}

\chapter{Conclusion and Future Directions}
\label{ch:conclu}
\section{Conclusion}
This thesis proposal focuses on the design and optimization on permissioned blockchains with database techniques. 
In light of the growing demand on blockchains as emerging transaction platforms, 
our mission is to identify their bottlenecks and pain points for the improvement while preserving the security.
This leads to a series of three works, the first as a behavior study, the second as a utility enhancement, and the last as the performance speedup. 
Throughout, we adopt the modularized methodology and ground all our implementation into {\fs}. 

Firstly, we presented a comprehensive dichotomy between blockchains and distributed databases, viewing them as two different types of transactional distributed systems. We proposed a taxonomy consisting of four design dimensions: replication, concurrency, storage, and sharding. Using this taxonomy, we discussed how both system types make different design choices driven by their high-level goals (e.g., security for blockchains, and performance for databases). We then performed a quantitative performance comparison using five different systems covering a large area of the design space. Our results illustrated the effects of different design choices to the overall performance. Our work provides the first framework to explore future database-blockchain design fusions. 

In second work, we showed how to build a fine-grained,
secure and efficient provenance system on top of blockchains. 
We implemented our techniques into {\fs}.
The system efficiently captures provenance information during runtime and stores it in secure storage. 
It exposes simple APIs to smart contracts, which enables a new class of provenance-dependent blockchain applications. 
Provenance queries are efficient in {\fs}, thanks to a novel skip
list index. We benchmarked it against several baselines. The results show the benefits of {\fs} in supporting rich,
provenance-dependent applications. We demonstrate that
provenance queries are efficient and that the system incurs
small storage and performance overhead.

Last but not the least, we proposed a novel solution to efficiently reduce the transaction abort rate in execute-order-validate blockchains by applying transactional
analysis from optimistic-concurrency-control databases. We first draw theoretical parallelism between both blockchains and databases. Then, we introduced a fine-grained concurrency control method.
In the later thesis, we will implement it in {\fs} and empirically evaluate its performance. 
% Our experimental analysis shows that {\fs} outperforms other blockchain systems, including the vanilla Fabric,
% and {\fabricPlusplus}. Unlike databases that achieve high throughput, the blockchains’ limited throughput due to factors related to security opens up opportunities for precise transaction management.

\section{Future Directions}
\subsection{Blockchain Interoperability}
While a growing number of blockchains proliferate, most of them operate in silos, with poor synchronization and coordination. 
Such fragmentation of the landscape not only results into a waste of resources and data isolation, but it also runs counter to the very essence of the Internet, openness and freedom. 
Even though we observe a number of research works with the special emphasis on the across-ledger token swap, their scope is mostly restricted to the cryptocurrency domain~\cite{herlihy2018atomic,robinson2019atomic,zakhary2019atomic}. 
For wider applications, the community of blockchains should look forward to some more generic standards, just like TCP/IP to the Internet. 
Promisingly, we notice that Interledger Protocol has taken on the initial attempt~\cite{interledger}. And we expect more will follow in the future. 

\subsection{Declarative Language for Smart Contracts}
Even though we demonstrate the provenance support on blockchains in Chapter~\ref{ch:prov}, it still follows an imperative approach.
To be specific, users are required to explicitly program \textit{how-to-do}, instead of implicitly declaring \textit{what-to-do} in the smart contract. 
The high-level declarative language can not only allow users to work on a high abstraction level and save development efforts. 
It also opens up a vast room for the common optimization. 
We haven't observed any progress along with this direction.
But according to the development roadmap of the database over the decades, we believe that an easy-to-use and intuitive contract scheme is essential for the mass adoption of blockchains. 

\subsection{Blockchain-like Verifiable Databases}
The impact of blockchains comes from its revolutionary decentralization.
But in reality, their byzantine tolerant consensus proves to be an overkill for most applications. 
In light of this, there are a growing number of secure databases, which, unlike blockchains, completely eliminate the decentralized setup~\cite{arasu2017concerto,zhang2020spitz}. Despite the single point, these databases still support verifiability on the state storage. Some vendors simulate a ledger-structure to expose the data provenance with the integrity guarantee~\cite{qldb}. 
We believe such blockchain-like verifiable databases already satisfy the majority of business requirements, in which blockchains would prove redundant. 

\subsection{Federated Learning on Blockchains}
Considering their common decentralized nature of the federated learning and blockchains, it is not hard to image a number of literatures that pair both hot topics together~\cite{lu2019blockchain,kim2019blockchained,awan2019poster}. 
The researchers have attempted to rely on blockchains to consolidate data from mutual distrusted users and collectively train for a shared model. 
In their design, a blockchain serves a trust-building platform to regulate on data ownership and the model copyright. 
But the challenge is that data privacy may be at odds with the blockchain transparency. 
And it remains a open topic how to fairly allocate the model ownership according to the heterogeneous data sources and use blockchains to coordinate this process. 
In the AI-driven future with the immense adoption of Internet-of-Things, we expect more such interdisciplinary proposals between blockchains and machine learning. 
 